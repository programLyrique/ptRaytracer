\subsection{La réflection}
Nous avons implémenté la réflection de la lumière sur de multiples sphères avant d'accéder à la source de lumière pour pouvoir ajouter un effet miroir aux sphères. Pour cela, nous avons considéré que le rayon se réfléchissait parfaitement sur la sphère avant de continuer. Cela permet de calculer l'éclairement par récurrence : la lumière arrivant en un point est celle que l'on recevrait avec une caméra placée en ce point et avec une direction celle du vecteur réfléchi \ref{reflection}.

L'implémentation a donné des résultats comme celui-ci \ref{ereflection}.

\subsection{La transparence}
  La transparence des objets a été aussi implémentée : la possibilité de voir un objet à travers un autre. Pour cela, un rayon peut traverser un objet, ce qui entraîne une diminution d'intensité. Les lois de Snell-Descartes ont ensuite été ajoutées, pour avoir la réfraction dans les sphères et gérer des effets comme celui de la lumière qui traverse une boule en verre \ref{refraction}.
Les résultats sont assez réalistes :  \ref{erefraction}.
