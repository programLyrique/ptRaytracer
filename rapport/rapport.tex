\documentclass[10pt,a4paper]{article}
\usepackage[utf8]{inputenc}


\usepackage[francais]{babel}
\usepackage[T1]{fontenc}
\usepackage{amsmath}
\usepackage{amsfonts}
\usepackage{amssymb}
\usepackage{graphicx}
\usepackage{lmodern}
\usepackage{multicol}
\usepackage{tikz}

\author{Pierre \bsc{Donat-Bouillud} Thibaud \bsc{Ehret}}
\title{Lancer de rayons}

% Parler surtout de l'architecture : 
% « Je n'attends pas de vous des démonstrations mathématiques et algorithmiques, mais une réflexion sur 
% les structures de données utilisée, le découpage en classes, les difficultés d'implémentations et les résultats. »

%% ATTENTION : 5 pages max de texte (toutes les images -> en annexe )

\begin{document}

\maketitle

\section*{Introduction} %Situer le problème du lancer de rayon

\section{Organisation du projet} % Parler ici de la hiérarchie de classes et des structures de données, de git, de github, et de Doxygen

\section{Illumination} % Parler des difficultés d'implémentations et des résultats visuels

\section{Améliorations} % Parler ici du multithreading, des réflexions et transparence, et de l'anticrénelage

\section*{Conclusion}


\tableofcontents

\appendix % Mettre ici toutes les images (et y faire réfèrence via \label{truc}, puis \ref{truc}



\end{document}