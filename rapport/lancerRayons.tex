\subsection{Les buts et les difficultés}
Nous avons cherché à implémenter une méthode permettant d'avoir un rendu réaliste, tout en n'étant pas trop compliqué à implémenter et permettant de faire certaines améliorations comme rajouter de la transparence ou la réflection des rayons sur les sphères. C'est pour cela que nous avons choisi d'implémenter l'illumination de Phong qui regroupait tout ce que nous cherchions.

\subsection{L'illumination de Phong}

L'illumination de Phong est composée de plusieurs parties: l'illumination ambiante, l'illumination spéculaire et l'illumination diffuse.
\begin{enumerate}
\item La lumière ambiante est constante dans l'espace et représente les parasites lumineux provenant de tous les points de l'espace. Nous l'avons considéré comme nulle dans notre projet, celle-ci étant en général gérée par des techniques telle que photon-mapping que nous n'avons pas eu le temps d'implémenter.
\item La lumière spéculaire est la lumière qui a réfléchi sur l'objet avant de parvenir à la caméra. on peut considérer une réflexion plus ou moins parfaite grâce à un facteur de brillance $\alpha$. En considérant les vecteurs unitaires comme définis sur le schéma \ref{speculaire}, l'intensité de la lumière parvenant à la caméra est alors $<\vec{R}|\vec{V}>^{\alpha} i_{s} k_{s}$ où $i_{s}$ l'intensité de la lumière incidente et $k_{s}$ une constante liée au matériau. En pratique, nous avons appliqué cette formule pour chaque canaux RGB et choisi les intensités comme les valeurs des canaux RGB de la lumière incidente et $k_{s}$ proportionnel au canal de la couleur naturel de l'objet.
\item La lumière diffuse correspond à la partie de la lumière qui est absorbée par l'objet avant d'être réémise isotropement et uniformement dans l'espace. En considérant les mêmes notations que sur le schéma \ref{diffuse}, l'intensité réémis est alors $i_{d}k_{d}<\vec{L}|\vec{N}>$ avec les mêmes remarques que pour la lumière spéculaire.
\end{enumerate}

Les sources de lumière multiples sont gérées de la manière suivante: pour chaque source, on fait le calcul de l'intensité reçue puis on somme le tout (pour rester dans les canaux RGB, l'intensité reçue possible est majorée dans notre implémentation).

Les résultats après implémentation sont les suivants \ref{phong}