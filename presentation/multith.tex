\begin{frame}[fragile]
\frametitle{Threads}

\begin{block}{Posix threads}
Une bibliothèque de threads C : créer un \emph{wrapper} C++ pour être plus idiomatique.
\end{block}

\begin{block}{Utilisation du \emph{wrapper}}
Hériter de la classe Thread, et implémenter :
\begin{verbatim}
virtual void run() = 0;
\end{verbatim}
\end{block}

\end{frame}

\begin{frame}[fragile]
\frametitle{\emph{Wrapper}}
\begin{verbatim}
class Thread
{
    public:
        Thread();
        virtual ~Thread();
        pthread_t getHandle() const { return thread; }
        bool exec();
        bool join();
        static unsigned int nbCores();
    protected:
        virtual void run() = 0;
    private:
        pthread_t thread;
        static void * startRoutine(void * obj);
};
\end{verbatim}
\end{frame}


\begin{frame}
\frametitle{Parallélisation du lancer de rayon}

\begin{block}{Facilement parallélisable}
Chaque pixel est rendu indépendamment des autres.
\end{block}

\begin{block}{Paralléliser}
\begin{itemize}
\item Détecter le nombre de coeurs.
\item Calcul du découpage\footnote{Oui, une puissance de 2 pour l'instant.} de l'image : $n \times m $ où $ n = \log_2(nbThreads) $ et $ m = nbThreads / n$.
\item Lancer un \emph{thread} par partie de l'image.
\end{itemize}
\end{block}

\end{frame}

\begin{frame}
\frametitle{Performances}
\begin{tabular}{|c|c|c|c|c|}
\hline 
Nombre de threads & 1 & 2 & 4 & 8 \\ 
\hline 
Temps (s) & 91 & 47 & 35 & 23 \\ 
\hline 
\end{tabular} 
\begin{block}{Banc de mesure}
\begin{tikzpicture}
\draw plot file {benchmark.txt};;
\end{tikzpicture}
\end{block}

\end{frame}

\begin{frame}
\frametitle{Améliorations}

\begin{itemize}
\item Découper l'image de façon adaptative.
\item Problèmes de synchronisation entre threads, et de conflit sur les données.
\end{itemize}
\end{frame}
