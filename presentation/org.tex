\begin{frame}
\frametitle{Organisation pratique}

\only<1>{
\begin{block}{Git et github}
	\begin{multicols}{2}
	\begin{itemize}
	\item versions
	\item système « d'issues » de Github
	\item nombreux graphiques pour visualiser l'avancement du travail
	\end{itemize}
	\begin{figure}[H]
	\begin{center}
  		\includegraphics[scale=0.15]{Github.png}		
	\end{center}
	\caption{Page d'accueil du projet.} \label{Github}
	\end{figure}

  	\end{multicols}
\end{block}
}

\uncover<2->{
  \begin{block}{Branches}
  \only<2>{
  	\begin{figure}
  	\begin{center}
  		\includegraphics[scale=0.3]{arbreBranchesGit.png} 	
  	\end{center}
  	\caption{Les branches se séparent.} \label{Separent}
  	\end{figure}
	}
	\uncover<3>{	
	\begin{figure}
	\begin{center}
		  \includegraphics[scale=0.3]{arbreBranchesGit3.png}
	\end{center}
	\caption{Et se rejoignent.} \label{Rejoindre}
	\end{figure}
	}
  \end{block}
}
\end{frame}

\begin{frame}{Principe du lancer de rayons}

\begin{itemize}
\item Lancer un rayon depuis la caméra, passant par un pixel de l'écran \footnote{Certaines techniques de lancer de rayon demandent un lancer de rayon depuis la caméra.}
\item Renvoyer le rayon vers les sources de lumière, en prenant en compte les éventuels réflexions, et réfractions.
\item Prendre en compte la texture de l'objet (couleur, rugosité).
\item Combiner les informations de couleur.
\end{itemize}

\begin{center}
\includegraphics[scale=0.5]{raytracing.png}
\end{center}
\end{frame}



\begin{frame}
\frametitle{Hiérarchie de classes}

\begin{center}
\includegraphics[scale=0.3]{hierarchie.png}
\end{center}

\end{frame}


\begin{frame}[fragile]
\frametitle{Structures de données}

\begin{verbatim}
std::vector<Mesh*> objets;
std::vector<Light*> lights;

\end{verbatim}


\uncover<2->
{
	\begin{alertblock}{Inconvénients}
	Lors de la recherche d'une intersection, complexité linéaire en le nombre d'objets.
	Idem pour les lampes.
	\end{alertblock}
}

\uncover<3->
{
	\begin{block}{Des structures de données plus efficaces}
	Utiliser des octrees pour subdiviser l'espace, et effectuer l'intersection avec des noeuds ( « bounding boxes »), de l'arbre.
	\end{block}
}

\end{frame}

